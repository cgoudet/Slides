% Created 2017-03-06 lun. 16:08
% Intended LaTeX compiler: pdflatex
\documentclass[bigger]{beamer}
\usepackage[utf8]{inputenc}
\usepackage[T1]{fontenc}
\usepackage{graphicx}
\usepackage{grffile}
\usepackage{longtable}
\usepackage{wrapfig}
\usepackage{rotating}
\usepackage[normalem]{ulem}
\usepackage{amsmath}
\usepackage{textcomp}
\usepackage{amssymb}
\usepackage{capt-of}
\usepackage{hyperref}
\usetheme{Boadilla}
\author{Christophe Goudet}
\date{\today}
\title{Discussion on calibration systematic model}
\beamertemplatenavigationsymbolsempty
\usepackage{appendixnumberbeamer}
\hypersetup{
 pdfauthor={Christophe Goudet},
 pdftitle={Discussion on calibration systematic model},
 pdfkeywords={},
 pdfsubject={},
 pdfcreator={Emacs 24.5.1 (Org mode 9.0.4)},
 pdflang={English}}
\begin{document}

\maketitle



\section{Correlation model}
\label{sec:orgd8ba212}

\begin{frame}[label={sec:org6c9537b}]{Introduction}
\begin{itemize}
\item Calibration uncertainties include \textasciitilde{}85 sources
\item Minimal model (ALL)
\begin{itemize}
\item Conservative (\(\Delta\) m\(_{\text{H}}\)=0.47\%)
\item Over-constrained
\end{itemize}
\item Maximal model has too many NP
\begin{itemize}
\item Realistic (\(\Delta\) m\(_{\text{H}}\)=0.27\%)
\item Too many parameters
\end{itemize}

\item Need a middle ground
\begin{itemize}
\item Realistic
\item Few parameters
\item Easy to combine
\end{itemize}

\item Status of systematics :
\begin{itemize}
\item \url{https://indico.cern.ch/event/607537/contributions/2453161/attachments/1401546/2139318/Couplings.pdf}
\item \url{https://indico.cern.ch/event/613697/contributions/2475511/attachments/1411833/2159897/manzoni\_13\_2\_2017.pdf}
\end{itemize}
\end{itemize}
\end{frame}

\section{Merging}
\label{sec:org85b5ac0}
\begin{frame}[label={sec:orga6f259f}]{Model Pre-fit}
Merge a selection of NP at the tool level (similar to current ALL model).
Model naively thought to be used when reading \href{https://cds.cern.ch/record/1637535/files/ATL-COM-PHYS-2013-1654\_2.pdf}{ATL-COM-PHYS-2013-1654}.
\vfill
\begin{columns}
\begin{column}{0.5\columnwidth}
Pros :
\begin{itemize}
\item Easy to use for analyses (change tool option)
\item Unique definition of NP for combination (\(\gamma \gamma\) and 4l)
\end{itemize}

Cons :
\begin{itemize}
\item Increase the total uncertainty
\item Must find a model which minimise the increase for all analyses
\end{itemize}
\end{column}

\begin{column}{0.5\columnwidth}
\begin{center}
\includegraphics[width=.9\linewidth]{/home/goudet/Documents/LAL/Zim/Hgam/PhotonSystematic/170209_CompareSystModel_h013_mean.pdf}
\end{center}

\begin{itemize}
\item FULLMerge : merging a set of NP
\item FULLNoEta : merging all dependencies in \(\eta\)
\end{itemize}
\end{column}
\end{columns}
\end{frame}

\begin{frame}[label={sec:org19271cb}]{Model Post-fit}
Merge a selection of NP after evaluation of effect on mass distribution.
Done in run 1 : \href{https://cds.cern.ch/record/1642851/files/ATL-COM-PHYS-2014-018.pdf}{ATL-COM-PHYS-2014-018}.
v\fill
Pros :
\begin{itemize}
\item Inclusive uncertainty strictly equals the full model (categorised uncertainty less trivial)
\item Optimisation of merging can be adapted in each analysis
\end{itemize}

Cons :
\begin{itemize}
\item Recommandations must be provided and followed for combination.
\item Does not reduce statistical fluctuations impact on uncertainties.
\end{itemize}
\end{frame}

\begin{frame}[label={sec:orgb67d4b1}]{Diagonalisation}
\begin{itemize}
\item Diagonalise covariance matrix and define new NP as eigen-vectors
\item Keep N higher eigen-values and merge remaining ones (as either model 1 or 2)
\end{itemize}
\vfill
Pros :
\begin{itemize}
\item Merged parameters are the ones having the less impact
\end{itemize}

Cons :
\begin{itemize}
\item Loss of physical meaning of NP
\end{itemize}
\end{frame}


\begin{frame}[label={sec:org4542a2a}]{Conclusion}
\begin{itemize}
\item h015 calibration systematic samples appearing
\item Need to define a strategy for HGam analyses (and H4l)
\begin{itemize}
\item Physical or eigen-vectors NP
\item Merging pre- or post- mass distribution fit
\end{itemize}
\end{itemize}
\end{frame}
\end{document}